%%%%%%%%%%%%%%%%%%%%%%%%%%%%%%%%%%%%%%%%%
% Beamer Presentation
% LaTeX Template
% Version 1.0 (10/11/12)
%
% This template has been downloaded from:
% http://www.LaTeXTemplates.com
%
% License:
% CC BY-NC-SA 3.0 (http://creativecommons.org/licenses/by-nc-sa/3.0/)
%
%%%%%%%%%%%%%%%%%%%%%%%%%%%%%%%%%%%%%%%%%

%----------------------------------------------------------------------------------------
%	PACKAGES AND THEMES
%----------------------------------------------------------------------------------------

\documentclass{beamer}

\mode<presentation> {

% The Beamer class comes with a number of default slide themes
% which change the colors and layouts of slides. Below this is a list
% of all the themes, uncomment each in turn to see what they look like.

%\usetheme{default}
%\usetheme{AnnArbor}
%\usetheme{Antibes}
%\usetheme{Bergen}
%\usetheme{Berkeley}
%\usetheme{Berlin}
%\usetheme{Boadilla}
%\usetheme{CambridgeUS}
%\usetheme{Copenhagen}
%\usetheme{Darmstadt}
%\usetheme{Dresden}
%\usetheme{Frankfurt}
%\usetheme{Goettingen}
%\usetheme{Hannover}
%\usetheme{Ilmenau}
%\usetheme{JuanLesPins}
%\usetheme{Luebeck}
\usetheme{Madrid}
%\usetheme{Malmoe}
%\usetheme{Marburg}
%\usetheme{Montpellier}
%\usetheme{PaloAlto}
%\usetheme{Pittsburgh}
%\usetheme{Rochester}
%\usetheme{Singapore}
%\usetheme{Szeged}
%\usetheme{Warsaw}

% As well as themes, the Beamer class has a number of color themes
% for any slide theme. Uncomment each of these in turn to see how it
% changes the colors of your current slide theme.

%\usecolortheme{albatross}
%\usecolortheme{beaver}
%\usecolortheme{beetle}
%\usecolortheme{crane}
%\usecolortheme{dolphin}
%\usecolortheme{dove}
%\usecolortheme{fly}
%\usecolortheme{lily}
%\usecolortheme{orchid}
%\usecolortheme{rose}
%\usecolortheme{seagull}
%\usecolortheme{seahorse}
%\usecolortheme{whale}
%\usecolortheme{wolverine}

%\setbeamertemplate{footline} % To remove the footer line in all slides uncomment this line
%\setbeamertemplate{footline}[page number] % To replace the footer line in all slides with a simple slide count uncomment this line

%\setbeamertemplate{navigation symbols}{} % To remove the navigation symbols from the bottom of all slides uncomment this line
}

\usepackage[utf8]{inputenc}
\usepackage{ragged2e}
\usepackage[export]{adjustbox}
\usepackage[flushleft]{threeparttable}
\usepackage{graphicx} % Allows including images
\usepackage{booktabs} % Allows the use of \toprule, \midrule and \bottomrule in tables
\usepackage[portuguese]{babel}
\usepackage{adjustbox}
\usepackage{graphicx}
\usepackage{multicol}
\usepackage{subfig}

%gets rid of bottom navigation bars
%\setbeamertemplate{footline}[frame number]{}

%gets rid of bottom navigation symbols
\setbeamertemplate{navigation symbols}{}

%gets rid of footer
%will override 'frame number' instruction above
%comment out to revert to previous/default definitions
%\setbeamertemplate{footline}{}

%\addtobeamertemplate{navigation symbols}{}{%
%    \usebeamerfont{footline}%
%    \usebeamercolor[fg]{footline}%
%    \hspace{1em}%
%    \insertframenumber/\inserttotalframenumber
%}
%
%\setbeamercolor{footline}{fg=blue}
%\setbeamerfont{footline}{series=\bfseries}

%----------------------------------------------------------------------------------------
%	TITLE PAGE
%----------------------------------------------------------------------------------------

\title[Defesa de Mestrado]{Caracterização de eventos de exceção e de seus respectivos impactos no sistema de transporte público por ônibus da cidade de São Paulo} % The short title appears at the bottom of every slide, the full title is only on the title page

\author[DIAS, F.; CORDEIRO, D.]{Felipe Cordeiro Alves Dias\\[1mm]Orientador: Prof. Dr. Daniel de Angelis Cordeiro}
\institute[USP-EACH] % Your institution as it will appear on the bottom of every slide, may be shorthand to save space
{
Universidade de São Paulo \\ % Your institution for the title page
\medskip
}
\date{\today} % Date, can be changed to a custom date

\begin{document}

\begin{frame}
\titlepage % Print the title page as the first slide
\end{frame}
%----------------------------------------------------------------------------------------
%	PRESENTATION SLIDES
%----------------------------------------------------------------------------------------

%------------------------------------------------
\section{Introdução}
\begin{frame}
\Huge{\centerline{Introdução}}
\end{frame}
%------------------------------------------------
\subsection{Motivação}
\begin{frame}
\frametitle{Motivação}
\begin{itemize}
\item Segregação urbana: dentre os mais de 12 milhões de habitantes da cidade de São Paulo, \alert{10\% estão localizados no Centro Expandido (CE)} e \alert{90\% no Cinturão Periférico (CP)}.
\begin{itemize}
\item Problemas relacionados a mobilidade urbana.
\end{itemize}
\end{itemize}

\begin{itemize}
\item Legislação federal e municipal sobre mobilidade urbana.
\begin{itemize}
\item Lei Federal 12.587/2012: \alert{para desenvolvimento sustentável com a mitigação dos custos ambientais e socioeconômicos dos deslocamentos de pessoas}.
\item Decreto 56.834: institui o \textit{PlanMob/SP 2015} \alert{como instrumento de planejamento e gestão do Sistema Municipal de Mobilidade Urbana para os próximos 15 anos}.
\end{itemize}
\end{itemize}

\end{frame}

%------------------------------------------------
\begin{frame}
\frametitle{Motivação}
\begin{itemize}
\item \textit{PlanMob/SP 2015}
\begin{itemize}
\item Criação da Central Integrada de Mobilidade Urbana (CIMU): com o objetivo de \alert{integrar as áreas de mobilidade urbana subordinadas à Secretaria Municipal de Transportes (SMT)}.
\item \alert{A CIMU não processa conteúdo de Redes Sociais};
\item \alert{não aborda melhorias dos sistemas já existentes};
\item \alert{será integrada com o defasado SIM}.
\end{itemize}
\end{itemize}

\begin{itemize}
\item Sistema Integrado de Monitoramento e Transporte (SIM).
\item Sistemas de Transporte Inteligente (ITS --- \textit{Intelligent Transport System}).
%: localização e rastreio dos ônibus, fornece informações em tempo real aos passageiros, monitora 1.353 rotas de ônibus, 10 corredores de ônibus, 28 terminais de ônibus e 19.933 mil paradas de ônibus que serviram em 2016 a aproximadamente 8 milhões de passageiros por dia. \item Apesar da importância do SIM, há inúmeras defasagens tecnológicas (que causam discrepância nas informações recebidas pelos usuários, dentre outros problemas).
\end{itemize}
\end{frame}
%------------------------------------------------
\begin{frame}
\frametitle{Motivação}
\begin{itemize}
\item A lei de mobilidade urbana (12.587/2012) e o \textit{PlanMob/SP 2015} \alert{não mencionam explicitamente ITS e TIC}.
\end{itemize}
\begin{itemize}
\item O transporte público \alert{pode se beneficiar ao integrar as Redes Sociais com o planejamento, gestão e as atividades operacionais dos transportes públicos}, abordando seus respectivos fatores sócio-técnicos, como:
\begin{itemize}
\item \alert{Analisar o impacto dos eventos de exceção na operação do sistema de transporte público por ônibus na cidade de São Paulo}.
\end{itemize}
\end{itemize}
\end{frame}
%------------------------------------------------
\subsection{Definição do problema}
\begin{frame}
\frametitle{Definição do problema}
%\begin{itemize}
%\item Eventos de exceção: acidentes, greves, falhas na operação do metrô, manifestações, enchentes, eventos sociais, dentre outros.
%\end{itemize}

\begin{itemize}
\item \alert{Problemas que envolvem a caracterização dos impactos dos eventos de exceção}:
\begin{itemize}
\item Corpus SPTrans.
\begin{itemize}
\item Processamento de grandes volumes de dados;
\item identificação de padrões;
\item qualidade dos dados comprometida.
\end{itemize}
\item Corpus Twitter.
\begin{itemize}
\item Identificação dos eventos de exceção nas publicações; 
\item extração de endereços e geolocalização; 
\item correlação de eventos de exceção com a base histórica.
\end{itemize}
\end{itemize}
\end{itemize}

%\begin{itemize}
%\item Uso de tais características para classificação de eventos de exceção.
%\end{itemize}

\end{frame}
%------------------------------------------------
\subsection{Objetivos}
\begin{frame}
\frametitle{Objetivos}
\begin{block}{Gerais}
Caracterização de eventos de exceção e de seus respectivos impactos no sistema de transporte público por ônibus da cidade de São Paulo.
\end{block}
\begin{block}{Específicos}
\begin{itemize}
    \item Identificar os eventos de exceção, quando existentes, dos \textit{tweets} coletados (\alert{classificação automatizada}).
     \item Extrair os endereços dos eventos de exceção identificados e geolocalizá-los.
		%\item Construir uma base de dados pública com os dados processados, disponibilizada via API (para consumo e contribuição da comunidade de software), mantendo o modelo de dados consistente.
\item Criação de plataforma para exploração e visualização dos dados coletados e processados do Twitter e da SPTrans.
\end{itemize}
\end{block}
\end{frame}
%------------------------------------------------
\begin{frame}
\frametitle{Hipóteses}

\begin{itemize}

\item É possível identificar e categorizar os eventos de exceção de acordo com os tipos de eventos encontrados pela Revisão Sistemática.

\item Extração de características com o auxílio de Processamento de Linguagem Natural (NLP --- \textit{Natural Language Processing}) em conjunto com dicionários auxiliares para o contexto dos eventos de exceção mencionados.

\item Extração dos endereços dos \textit{tweets} utilizando a técnica de Expressão Regular e posterior geolocalização.
\end{itemize}

\end{frame}
%------------------------------------------------
\section{Fundamentação Teórica}
\begin{frame}
\Huge{\centerline{Fundamentação Teórica}}
\end{frame}
%------------------------------------------------
\subsection{Cidades Inteligentes}
\begin{frame}
\frametitle{Cidades Inteligentes}
\begin{block}{Definição}
São cidades sustentáveis e socialmente inclusivas, que utilizam Tecnologias da Informação e Comunicação (TICs) para gerir eficientemente seus respectivos recursos naturais.
\begin{itemize}
\item Método com viés tecnológico.
\item Método com viés humano.
\end{itemize}
\end{block}

%\begin{block}{Cidade}
%Complexo e dinâmico sistema sócio-técnico. Composto por sistemas urbanos, com espaços físicos para a vida cotidiana e com sistemas de infraestrutura.
%\end{block}
\end{frame}
%------------------------------------------------
\begin{frame}
\frametitle{Cidades Inteligentes}
\begin{itemize}
\item As TICs permeiam os sistemas urbanos e espaços físicos: dados voluntários, de sensores e Redes Sociais.
\item Desafios relacionados a conectividade:
\begin{itemize}
\item Infraestrutura de rede.
\item interoperabilidade;
\item padrões;
\item consumo de energia;
\item escalabilidade, dentre outros.
\end{itemize}
\item Desafios relacionados aos dados:
\begin{itemize}
\item Capacidade e local de armazenamento;
\item extração;
\item tratamento;
\item processamento;
\item análise;
\item integração;
\item agregação de dados, dentre outros.
\end{itemize}
\end{itemize}
\end{frame}
%------------------------------------------------
\subsection{Sistemas de Transporte Inteligente}
\begin{frame}
\frametitle{Sistemas de Transporte Inteligente}
\begin{block}{Definição}
Tem como fim utilizar TICs para resolver problemas relacionados ao transporte, tais como congestionamento, segurança, eficiência e conservação ambiental. 
\end{block}
\begin{itemize}
%\item \textbf{Operação de Veículos Comerciais} (CVO --- \textbf{\textit{Commercial Vehicles Operation})} --- são sistemas utilizados para a segurança de veículos comerciais e frotas, por meio de tecnologias relacionadas a gerenciamento de tráfego, controle e gerenciamento de veículos e informações aos viajantes, tais como:
%\begin{itemize}
\item Localização de Veículos Autônomos (\textit{Automatic Vehicles Location}).
%\end{itemize}
\end{itemize}
\end{frame}
%------------------------------------------------
%\subsection{Conceitos relacionados ao transporte público}
%\begin{frame}
%\frametitle{Conceitos relacionados ao transporte público}
%\begin{itemize}
%\item Acessibilidade.
%\item Acessibilidade universal.
%\item Mobilidade.
%\item Mobilidade urbana.
%\begin{itemize}
%\item Transporte público coletivo;
%\item  transporte de alta capacidade;
%\item  acessibilidade universal nos passeios e edificações;
%\item prioridade ao transporte coletivo no sistema viário;
%\item terminais de transporte intermodais;
%\item rede de transporte coletivo por ônibus (com acessibilidade universal);
%\item rede cicloviária;
%\item bicicletários e paraciclos;
%\item  legibilidade dos sistemas de orientação;
%\item comunicação eficaz com os usuários;
%\item modicidade tarifária;
%\item  logística eficiente no transporte de carga, dentre outros itens.
%\end{itemize} 
%\end{itemize}
%\end{frame}
%------------------------------------------------
\subsection{General Transit Feed Specification}
\begin{frame}
\frametitle{General Transit Feed Specification}
\begin{block}{Definição}
É uma especificação de um formato comum para troca de informações estáticas sobre transporte público.
\begin{itemize}
\item \textit{agency.txt}: agências de transporte público como fonte de dados.
\item \textit{stops.txt}: locais de embarque e desembarque.
\item \textit{routes.txt}: trajetos de um grupo de viagens.
\item \textit{trips.txt}: viagens de cada trajeto.
\item \textit{stop\_times.txt}: horários de partida e chegada em paradas.
\item \textit{calendar.txt}: início, fim e dias disponíveis dos serviços.
\end{itemize}
\end{block}
\end{frame}
%------------------------------------------------
%\subsection{Redes Sociais}
%\begin{frame}
%\frametitle{Redes Sociais}
%\begin{block}{Redes Sociais}
%As Redes Sociais podem ser definidas como redes que possuem muitos relacionamentos, com grandes componentes conectados, altos coeficientes de agrupamento e grau de reciprocidade. Ex.: Facebook.
%\end{block}
%\begin{block}{Rede de Informações}
%Nesse tipo de rede a interação dominante é a disseminação de informações entre os relacionamentos, com baixo baixo índice de reciprocidade. Ex.: Twitter.
%\end{block}
%\end{frame}
%------------------------------------------------
\subsection{Processamento de Linguagem Natural}
\begin{frame}
\frametitle{Processamento de Linguagem Natural}
\begin{block}{Definição}
Explora como computadores podem ser utilizados para entender e manipular texto ou fala em linguagem natural, o que envolve conhecimento interdisciplinar.
% principalmente entre as áreas de ciência da computação, linguística e estatística.
\end{block}
\begin{multicols}{2}
\begin{itemize}
\item Problemas de baixo nível (comuns a NLP).
%\begin{itemize}
%\item Desambiguação do limite da sentença;
%\item \textit{Tokenização};
%\item Marcação de parte da fala;
%\item dentre outros.
%\end{itemize}
\end{itemize}

\columnbreak

\begin{itemize}
\item Problemas de alto nível (específicos e com base nos problema de baixo nível).
%\begin{itemize}
%\item Identificação e recuperação de erros ortográficos  e gramaticais;
%\item Identificação de entidade nomeada;
%\item Desambiguação do sentido da palavra;
%\item dentre outros.
%\end{itemize}
\end{itemize}
\end{multicols}
\end{frame}
%------------------------------------------------
\subsection{Feature Engineering}
\begin{frame}
\frametitle{Feature Engineering}
\begin{block}{Definição}
Processo iterativo de construção, extração e seleção de variáveis (features), o qual depende do conhecimento de domínio e de suas respectivas métricas.
\end{block}
\begin{itemize}
\item Variáveis (\textit{features}) binárias, categóricas ou contínuas.
\item Pré-processamento: técnicas de padronização, normalização, remoção de ruídos, redução de dimensionalidade, discretização, expansão, etc.
\end{itemize}
\end{frame}
%------------------------------------------------
\subsection{Algorítimos de Aprendizado de Máquina Supervisionados}
\begin{frame}
\frametitle {Algorítimos de Aprendizado de Máquina Supervisionados}
\begin{itemize}
\item Árvore de Decisão
\item Floresta Aleatória
\item K-ésimo Vizinho mais próximo
\item Máquina de Vetores de Suporte
\item Naive Bayes
\item Redes Neurais
\item Regressão Logísitica
\end{itemize}
\end{frame}
%------------------------------------------------
\subsection{TF-IDF}
\begin{frame}{TF-IDF}
\begin{block}{Definição}
TF-IDF é um algoritmo de ponderação de variáveis que combina as ponderações \emph{frequência do termo} (TF --- \textit{Term Frequency}) e \emph{inverso da frequência nos documentos} (IDF --- \textit{Inverse Document Frequency}) para calcular os pesos dos termos linguísticos (variáveis) em um determinado corpus.
\end{block}
\end{frame}
%------------------------------------------------
\subsection{Algoritimo Apriori}
\begin{frame}
\frametitle {Algoritimo Apriori}
%O algoritmo \textit{Apriori} normalmente é utilizado em mineração de texto para identificar relações entre conjuntos de itens e padrões, por meio de comparações de conjuntos de itens frequentes, para assim determinar regras de associação com base em métricas, tais como:
\begin{itemize}
\item \alert{\textit{Suporte} (\textit{support})}: indicador da frequência de determinados registros no conjunto de dados.
\item \alert{\textit{Confiança} (\textit{confidence})}: frequência com que determinadas regras de associações entre registros são encontradas como verdadeiras.
\item  \alert{\textit{Lift}}: probabilidade de ocorrência de um consequente B no conjunto de dados ($lift > 1$ indica que a regra de associação em questão pode ser utilizada para predição de um consequente B em conjuntos de dados futuros).
\end{itemize}

\alert{A notação $A \rightarrow B$ se refere a antecedente e consequente, respectivamente}.
 
\end{frame}
%------------------------------------------------
\section{Revisão Sistemática}
\begin{frame}
\Huge{\centerline{Revisão Sistemática}}
\end{frame}
%------------------------------------------------
\begin{frame}
\frametitle{Revisão Sistemática}
\begin{block}{Quais os tipos de problemas urbanos abordados utilizando processamentos de \textit{tweets}? (QP1)}
\begin{itemize}
\item \textit{e-Participation}.
\item Detecção de zoneamento urbano.
\item Identificação de pontos de interesse.
\item Mobilidade.
\item Padrões demográficos.
\item Poluição.
\item Segurança Pública.
\item Turismo.
\item Tráfego.
\end{itemize}
\end{block}
\end{frame}
%------------------------------------------------
\begin{frame}
\frametitle{Revisão Sistemática}
\begin{block}{Casos de uso relacionados ao transporte público (QP2)}
\begin{itemize}
\item Impacto de eventos no transporte público relacionados:
\begin{itemize}
\item \alert{Aos ataques terroristas em Paris no uso do transporte público}.
\item \alert{Ao tráfego na demanda por bicicletas, em Nova Iorque e Washington D.C}.
\item \alert{Aos pontos de interesse na demanda por transporte público}.
\item \alert{Aos eventos anormais nas tomadas de decisão dos passageiros do Metrô de Tóquio}.
\item \alert{A predição de fluxo de passageiros no Metrô de Nova Iorque}.
\end{itemize}

\item Planejamento e gestão do transporte público.
%\begin{itemize}
%\item Análise de sentimento relacionada ao acesso ao transporte público.
%\item Coleta de informações relacionadas ao transporte público.
%\item Identificação de locais para estações de bicicletas, em St. Petersburg, Rússia.
%\item Identificação da disposição dos usuários para realizar viagens de lazer.
%\item Plataforma para notificação de problemas relacionados ao transporte público de Bangalore, Índia.
%\end{itemize}
\end{itemize}
\end{block}

\end{frame}
%------------------------------------------------
\begin{frame}
\frametitle{Revisão Sistemática}
\begin{block}{Técnicas estatísticas utilizadas no processamento de \textit{tweets} (QP3)}
\begin{itemize}
\item \alert{Análise de métricas relacionadas a desempenho}.
\item Semelhança de cosseno.
\item \alert{\textit{${F_1}$ score}}.
\item \alert{Frequência do termo–inverso da frequência nos documentos (TF-IDF)}.
\item Coeficiente de variação inversa.
\item Método de reamostragem Jackknife.
%\item \textit{Linear Regression}.
\item Indicadores locais de associação espacial (LISA).
\item Local Moran's.
\item Máxima verossimilhança.
\item Média móvel integrada autoregressiva sazonal (SARIMA).
\item Otimização e previsão com função de perda híbrida.
\end{itemize}
\end{block}
\end{frame}
%------------------------------------------------
\begin{frame}
\frametitle{Revisão Sistemática}
\begin{block}{Paradigmas de processamento (QP4)}
\begin{itemize}
\item \textit{Processamento em lote} (offline).
\item \textit{Processamento em quase tempo real}.
\item \textit{Processamento em tempo real}.
\end{itemize}
\end{block}

\begin{block}{Eventos de exceção relacionados ao transporte público (QP5)}
\begin{itemize}
\item \alert{Acidentes}.
\begin{itemize}
\item Acidentes nas estações transporte.
\item Incêndio.
\end{itemize}

\item \alert{Espaço-temporais}.
\begin{itemize}
\item Dia da semana.
\item Hora do dia.
\end{itemize}

\end{itemize}
\end{block}

\end{frame}
%------------------------------------------------
\begin{frame}
\frametitle{Revisão Sistemática}
\begin{block}{Eventos de exceção relacionados ao transporte público (QP5)}
\begin{itemize}
\item \alert{Eventos sociais}.
\begin{itemize}
\item Feiras de rua.
\item Festivais.
\item Jogos esportivos.
\item Passeatas e maratonas.
\end{itemize}

\item \alert{Eventos urbanos}.
\begin{itemize}
\item Relacionados ao tráfego.
\end{itemize}

\item \alert{Desastres naturais}.
\begin{itemize}
\item Tempestades.
\item Terremoto.
\item Tufões.
\end{itemize}

\item \alert{Metereológicas}.
\begin{itemize}
\item Dia claro, nublado, chuvoso, nevando, com neblina.
\item Temperatura do ar.
\end{itemize}

\end{itemize}
\end{block}
\end{frame}
%------------------------------------------------
\begin{frame}
\frametitle{Revisão Sistemática}
\begin{block}{Técnicas de Aprendizado de Máquina utilizadas no processamento de \textit{tweets} (QP6)}
\begin{itemize}
\item \alert{Classificação \textit{bayesiana}}.
\item Algoritmo C5.0.
\item Campo aleatório condicional com Regressão Logística.
%\item \textit{Event extraction based on tweet hashtags}.
\item Alocação latente de Dirichle (LDA).
\item Regressão Linear.
\item Simulação Monte Carlo.
\item Técnica inovadora que utiliza fatorização tensorial   (\textit{PairFac}).
\item \alert{Floresta Aleatória}.
\item \alert{Máquina de Vetores de Suporte}.
\item Mapas auto-organizados.
\end{itemize}

\end{block}
\end{frame}
%------------------------------------------------
\section{Dados abertos relacionados ao transporte público e eventos de exceção}
\begin{frame}
\Huge{\centerline{Dados abertos relacionados ao}}
\Huge{\centerline{transporte público}}
\Huge{\centerline{e eventos de exceção}}
\end{frame}
%------------------------------------------------
\begin{frame}
\frametitle{Corpus SPTrans}
\begin{table}[!htb]
\centering
\caption{Arquivos e número de registros especificados na GTFS pela SPTrans}
	\label{tab:gtfs}
\begin{tabular}{c|c}
\toprule
\textbf{Nome do arquivo} & \textbf{Número de registros} \\ 
\midrule
\textit{agency.txt} & 1 \\ 
\hline
\textit{calendar.txt} & 6 \\ 
\hline
\textit{fare\_attributes.txt} & 6 \\ 
\hline
\textit{fare\_rules.txt} & 5.400 \\
\hline
\textit{frequencies.txt} & 39.625 \\
\hline
\textit{routes.txt} & 291.634 \\
\hline
\textit{shapes.txt} & 800.767 \\
\hline
\textit{stop\_times.txt} & 95.134 \\  
\hline
\textit{stops.txt} & 19.933 \\ 
\hline
\textit{trips.txt} & 2.273 \\
\midrule
\midrule
\textbf{\alert{Total}} & \alert{1.254.779} \\
\bottomrule
\end{tabular}
\end{table}
\end{frame}
%------------------------------------------------
\begin{frame}{Corpus SPTrans}
\begin{table}[!htb]
\begin{adjustbox}{max height=10mm, width=4in}
\begin{threeparttable}
\centering
\caption{Descrição do conjunto de dados AVL}
\label{tab:avlDataset}
\begin{tabular}{ c | c | c | c }
\toprule
\textbf{Mês} & \textbf{Intervalo (dias)} & \textbf{Total de arquivos AVL} & \textbf{Espaço em disco (GB)} \\
\midrule
Janeiro\tnote{a} & 1 - 31 & 744 & 102,44 \\
\hline
 Fevereiro & 1 - 28 & 672 & 93,21 \\
\hline
 Março & 1 - 31 & 744 & 102,64 \\
\hline
 Abril & 1 - 30 & 720 & 97,04 \\
\hline
 Maio & 1 - 31 & 744 & 101,46 \\
\hline
 Junho & 1 - 30 & 720 & 97,13 \\
\hline
 Julho & 1 - 31 & 744 & 104,95 \\
\hline
 Agosto & 1 - 31 & 744 & 108,38 \\
\hline
 Setembro & 1 - 30 & 720 & 109,89 \\
\hline
 Outubro & 1 - 31 & 744 & 110,92 \\
\hline
 Novembro & 1 - 30 & 717 & 108,16 \\
\hline
 Dezembro & 1 - 31 & 738 & 110,89 \\
\midrule
\midrule
\textbf{\alert{Total}} & --- & \alert{8.751} & \alert{1.247,09} \\
\bottomrule
\end{tabular}
\begin{tablenotes}
%\item[a]Arquivos indisponíveis em novembro: 
%\begin{itemize}
%\item \texttt{Movto\_201711011200\_201711011300.zip}
%\item \texttt{Movto\_201711011300\_201711011400.zip}
%\item \texttt{Movto\_201711011400\_201711011500.zip}
%\end{itemize}
%Justificativa ao recurso em primeira instância de acesso a informação: ``Conheço do recurso e nego provimento, informando que os registros solicitados não existem na nossa base e também não há informações de ``log'' que indiquem possíveis falhas ou indisponibilidade no sistema. Sendo assim, conforme já explicado anteriormente, não há condições técnicas de disponibilizar essas informações.'' --- resposta oficial da SPTrans, responsável: Paulo Cézar Shingai Diretor, Presidente da SPTrans.
%\item[b]Arquivos indisponíveis em dezembro, devido a falha na rede de transmissão de dados \nomenclature{GPRS}{\textit{General Packet Radio Services,}}conforme apresentado no sistema interno de registro de interrupções do sistema, Figura~\ref{fig:e_sic_33310} --- resposta oficial da SPTrans, responsável: Albino Silva da Rocha, Chefe de Gabinete da SPTrans: 
%\begin{itemize}
%\item \texttt{Movto\_201712150100\_201712150200.zip}
%\item \texttt{Movto\_201712150400\_201712150500.zip}
%\item \texttt{Movto\_201712150500\_201712150600.zip}
%\item \texttt{Movto\_201712150600\_201712150700.zip}
%\item \texttt{Movto\_201712150700\_201712150800.zip}
%\item \texttt{Movto\_201712150800\_201712150900.zip}
%\end{itemize}
\item[a] Arquivos  \texttt{Movto\_201701111000\_201701111100} com 35 campos na linha 60.025 e \texttt{Movto\_201701110900\_201701111000} com 21 campos na linha 1.075.548, o esperado são 19 campos de acordo com os metadados fornecidos pela SPTrans.
\end{tablenotes}
\end{threeparttable}
\end{adjustbox}
\end{table}
\end{frame}
%------------------------------------------------
\begin{frame}
\frametitle{Corpus Twitter}
\begin{table}[!htb]
\centering
\caption{Intervalo de tempo e número de \textit{tweets} coletados}
	\label{tab:tweetsCollected}
\begin{adjustbox}{max height=28mm}
\begin{threeparttable}
\begin{tabular}{c|c|c|c}
\toprule
\textbf {Perfil no \textit{Twitter}} &\textbf{Total de \textit{tweets}\tnote{a}}  &\textbf{ \textit{Timestamp 1\tnote{b}}} & \textbf{\textit{Timestamp 2\tnote{c}}} \\ 
\midrule
\textit{@BombeirosPMESP} & 6.632 & 2017-05-21 & 2017-12-01 \\
\hline
\textit{@CETSP\_} & 5.735 & 2017-02-20  & 2017-12-01 \\
\hline
\textit{@CPTM\_oficial} & 6.301 & 2017-04-24 & 2017-12-01 \\
\hline
\textit{@governosp}  & 6.011 & 2017-05-10 & 2017-12-01 \\
\hline
\textit{@metrosp\_oficial} & 8.621 & 2017-06-07 & 2017-12-01 \\
\hline
\textit{@Policia\_Civil}  & 3.417 & 2015-04-15 & 2017-11-30 \\
\hline
\textit{@PMESP}  & 4.365 & 2016-06-02 & 2017-11-30 \\
\hline
\textit{@saopaulo\_agora}  & 3.960 & 2016-11-18 & 2017-11-30 \\
\hline
\textit{@smtsp\_} & 1.316 & 2017-04-26 & 2017-12-01 \\
\hline
\textit{@SPCEDEC} & 1.301 & 2015-06-09 & 2017-12-01 \\
\hline
\textit{@sptrans\_} & 9.956 & 2017-06-13 & 2017-12-01 \\
\hline
\textit{@TurismoSaoPaulo} & 3.369 & 2012-06-12 & 2017-11-29 \\
\midrule
\midrule
\textbf{\alert{Total}} & \alert{60.984} & --- & --- \\
\bottomrule
\end{tabular}
\begin{tablenotes}
            \item[a] Número de \textit{tweets} coletados.
            \item[b] \textit{Timestamp} mais antigo.
            \item[c] \textit{Timestamp} mais recente.
        \end{tablenotes}
\end{threeparttable}
\end{adjustbox}
\end{table}
\end{frame}
\begin{frame}{Corpus Twitter}
Pré-processamento
\begin{itemize}
\item \textit{Case folding}: processamento de normalização de todas as letras do texto (de A-Z) para minúsculas.
%\item \textit{\textbf{Tokenization}}: processamento realizado para obtenção das palavras  (\textit{tokens}) que compõem uma sentença, inclui a remoção de números, pontuações e caracteres que não pertencem ao alfabeto \cite{Setiawan2017}.  
%\item \textbf{Remoção de} \textit{\textbf{stopwords}}: processamento para remoção do conjunto de \textit{tokens} de palavras sem significado ou importância \cite{Setiawan2017}, o que reduz a quantidade de ruído do conteúdo \textit{tweet} \cite{Steiger2015Census}.
%\item \textit{\textbf{Stemming}}: processamento para encontrar a raiz de uma palavra, removendo sufixos e prefixos (no caso do Português Brasileiro) das palavras derivadas \cite{Setiawan2017}.
\item Remoção de \textit{URLs} e menções a outros \textit{tweets}.
\item Remoção de acentos, \textit{emoticons} e pontuações substituídas por espaços vazios.
%\item Correção erros de digitação por meio de uma função de mapeamento
\item \textit{Stemming}. 
\end{itemize}

\end{frame}
%------------------------------------------------
\section{Correlação entre os tweets, dados AVL e GTFS da SPTrans}
\begin{frame}{Correlação entre os tweets, dados AVL e GTFS da SPTrans}
\begin{figure}
\includegraphics[width=0.9\linewidth]{caracterization_flow.png}
\end{figure}
\end{frame}
%------------------------------------------------
\section{Exploração e visualização de grandes volumes de dados}
\begin{frame}
\Huge{\centerline{Exploração e visualização}}
\Huge{\centerline{de grandes volumes de dados}}
\end{frame}
%------------------------------------------------
%\subsection{Correlação dos eventos de exceção com os dados AVL da SPTrans}
%\begin{frame}
%\frametitle{Correlação dos eventos de exceção com os dados AVL da SPTrans}
%\begin{itemize}
%\item Atraso médio induzido nas viagens.
%\item Ônibus frequentemente afetados por eventos de exceção.
%\item Ônibus frequentemente afetados por determinado evento de exceção.
%\item Padrão de ocorrência dos eventos de exceção no espaço-tempo (localizações e \textit{timestamps}).
%\item Quantidade e viagens afetadas.
%\item Quantidade e regiões da cidade de São Paulo afetadas.
%\item Viagens frequentemente afetadas por eventos de exceção.
%\item Viagens frequentemente afetadas por determinado evento de exceção.
%\end{itemize}
%\end{frame}
%------------------------------------------------
%\begin{frame}{Exploração e visualização de grandes volumes de dados}
%\begin{block}{Trabalhos relacionados}
%\begin{itemize}
%\item Apresentação de conceitos básicos e fluxos de visualização de dados de tráfego (dos dados brutos, pré-processamento ao mapeamento visual, construído com símbolos visuais). Técnicas e métodos de processamento de dados para descrever propriedades temporais, espaciais, numéricas e categóricas de dados de tráfego.
%\item Descrição de uma tipologia de dados de tráfego, capaz de abordar suas respectivas propriedades, problemas e transformações relevantes para a análise. Abordagens analíticas visuais para analisar dados de tráfego de veículos, pedestres, passageiros dentro de sistemas de transporte, etc.
%\item Apresentação de um novo algoritmo para mapeamento de medições coletivas para monitorar as infraestruturas de transporte terrestre e, aliviar o impacto de imprecisões do GPS para monitoramento contínuo de infraestruturas de transporte por meio de \textit{smart phones}.
%\end{itemize}
%\end{block}
%\end{frame}
%------------------------------------------------
%\begin{frame}{Exploração e visualização de grandes volumes de dados}
%\begin{block}{Definição}
%São cidades sustentáveis e socialmente inclusivas, que utilizam Tecnologias da Informação e Comunicação (TICs) para gerir eficientemente seus respectivos recursos naturais.
%\end{block}
%\begin{block}{Desafios}
%    \begin{itemize}
%        \item \alert{Conectividade:} Infraestrutura de redes, interoperabilidade, escalabilidade, tolerância a falhas, etc.
%        \item \alert{Data:} Capacidade de armazenamento e localização dos dados, extração, processamento, análise, exploração e visualização; correlação de dados de fontes heterogêneas, processamento em tempo real, etc.
%    \end{itemize}
%\end{block}
%\end{frame}
\subsection{Arquitetura proposta}
\begin{frame}
\frametitle{Arquitetura proposta}
%\begin{block}{Apache Superset}
%Aplicação web para exploração e visualização de dados.
%\end{block}
%\begin{block}{Apache Kafka}
%Aplicação para processamento de fluxos de dados em quase tempo real.
%\end{block}
\begin{figure}[!htb]% H manda colocar exatamente nessa posição no texto (relativa aos parágrafos anterior e posterior)
	\centering
 	  \caption{Arquitetura utilizada no estudo de caso}
		\includegraphics[width=1\linewidth]{viz_arch_pt.png}
	\label{fig:viz_arch}
%  \source{Felipe Cordeiro Alves Dias, 2017}
\end{figure}
\end{frame}
%------------------------------------------------
\begin{frame}
\frametitle{Arquitetura proposta}
\begin{block}{Druid}
\begin{itemize}
    \item O Druid é um banco de dados para análises exploratórias em tempo real (latências abaixo de sub-segundos) em grandes conjuntos de dados.
    \item Arquitetura distribuída composta por um cluster com diferentes tipos de nós (real-time, historical, broker e coordinator nodes).
    \item Nós independentementes uns dos outros e possuem interação mínima entre eles. 
%    \item Dependências externas: (I) Apache Zookeeper, responsável pela coordenação do cluster e (II) um sistema de gerenciamento de banco de dados relacional (RDBMS — Relational Data- base Management Systems), para armazenar parâmetros operacionais adicionais e configurações.
\end{itemize}
\end{block}
\end{frame}
%------------------------------------------------
\begin{frame}
\frametitle{Arquitetura proposta}
\framesubtitle{Real-time nodes}
\alert{Real-time nodes}
\begin{itemize}
\item Ingerir, indexar e consultar fluxos de eventos. 
\item Índices localmente persistidos são mesclados em \alert{blocos imutáveis} de dados com todos os eventos ingeridos em um período de tempo.
%\item Segmentos imutáveis: podem posteriormente serem carregados para uma camada de sistema de arquivos (deep storage).
\item \alert{Não há perda de dados e a imutabilidade dos blocos permite a consistência de leitura e um modelo de paralelização simples}: \textit{historical nodes} podem simultaneamente examinar e agregar blocos imutáveis de forma não bloqueante.
\end{itemize}
\end{frame}
\begin{frame}
\frametitle{Arquitetura proposta}
\framesubtitle{Historical, broker e coordinator nodes}
\begin{itemize}
    \item \alert{Historical nodes}: são responsáveis por carregar, descartar e servir segmentos imutáveis.
%     por meio de uma arquitetura shared-nothing (sem um único ponto de contenção entre os nós).
    \item\alert{Broker nodes}: são responsáveis por receber consultas e mesclar resultados parciais dos historicals e real-time nodes antes de retornar um resultado final consolidado para o cliente.
    \item \alert{Coordinator nodes}: são responsáveis pelo gerenciamento e distribuição dos dados nos historical nodes, exigindo destes o carregamento, descarte e replicação dos dados.
\end{itemize}
\end{frame}
%------------------------------------------------
\subsection{Validação da arquitetura proposta}
\begin{frame}
\frametitle{Validação da arquitetura proposta}
\framesubtitle{Quantidade de dados enviados por dia por ônibus (selecionados aleatoriamente) em janeiro de 2017}
\begin{figure}[!htb]% H manda colocar exatamente nessa posição no texto (relativa aos parágrafos anterior e posterior)
	\centering
		\includegraphics[width=0.90\linewidth]{analysis_by_bus_lines_pt.png}
	\label{fig:only_one_bus_map}
  %\source{Felipe Cordeiro Alves Dias, 2017}
\end{figure}
\end{frame}
\begin{frame}
\frametitle{Validação da arquitetura proposta}
\framesubtitle{Distribuição da quantidade de dados enviados por ônibus (selecionados aleatoriamente) em janeiro de 2017}
\begin{figure}[!htb]% H manda colocar exatamente nessa posição no texto (relativa aos parágrafos anterior e posterior)
	\centering
		\includegraphics[width=0.85\linewidth]{pizza_bus.png}
	\label{fig:buses_map}
\end{figure}
\end{frame}
%------------------------------------------------
\begin{frame}
\frametitle{Validação da arquitetura proposta}
\framesubtitle{Localizações enviadas em Janeiro de 2017 de uma linha de ônibus selecionada aleatoriamente}
\begin{figure}[!htb]% H manda colocar exatamente nessa posição no texto (relativa aos parágrafos anterior e posterior)
	\centering
		\includegraphics[width=0.85\linewidth]{only_one_bus_map.png}
	\label{fig:analysis_by_bus_lines}
  %\source{Felipe Cordeiro Alves Dias, 2017}
\end{figure}
\end{frame}
%------------------------------------------------
\begin{frame}
\frametitle{Validação da arquitetura proposta}
\framesubtitle{Localizações dos ônibus referente a movimentação de Janeiro de 2017}
\begin{figure}[!htb]% H manda colocar exatamente nessa posição no texto (relativa aos parágrafos anterior e posterior)
	\centering
		\includegraphics[width=0.90\linewidth]{buses_map.png}
	\label{fig:pizza_bus}
  %\source{Felipe Cordeiro Alves Dias, 2017}
\end{figure}
\end{frame}
%------------------------------------------------
\begin{frame}{Consideração sobre a arquitetura utilizada para exploração e visualização dos dados AVL da SPTrans}
\begin{itemize}
    \item Estudo de caso relacionado à visualização de grandes conjuntos de dados, utilizando dados dos ônibus da cidade de São Paulo. 
    \item Mostramos que é possível encontrar padrões complexos e incomuns e possíveis eventos de exceção em grandes conjuntos de dados por meio da visualização. 
    \item O Druid e o Apache Superset demonstraram suporte a agregação, exploração e visualização de grandes conjuntos de dados.
\end{itemize}
\end{frame}
%------------------------------------------------
\section{Mineração e geolocalização automatizada de eventos de exceção a partir de dados do Twitter}
\begin{frame}
\Huge{\centerline{Mineração e geolocalização}}
\Huge{\centerline{automatizada de eventos de exceção}}
\Huge{\centerline{a partir de dados do \textit{Twitter}}}
\end{frame}
%------------------------------------------------
\begin{frame}{Mineração e geolocalização automatizada de eventos de exceção a partir de dados do \textit{Twitter}}
    \begin{figure}[!htb]% H manda colocar exatamente nessa posição no texto (relativa aos parágrafos anterior e posterior)
	\centering
		\includegraphics[width=1\linewidth]{tweet_based_methodology_pt.png}
	\label{fig:pizza_bus}
  %\source{Felipe Cordeiro Alves Dias, 2017}
\end{figure}
Expressão regular para extração de endereços:
\begin{equation}
\resizebox{.9 \textwidth}{!} 
{
$ER = \lbrace L_1 | S_1 | L_2 | S_2 | \dots | L_n | S_n \rbrace \lbrace [a-z\grave{A}-\ddot{y}\_] + \rbrace$
}
\end{equation}
Geolocalização dos endereços usando a API do Google Geocoding.
\end{frame}
%------------------------------------------------
\begin{frame}{Resultados da classificação, manual, pré-processamento, processamento dos \textit{tweets} e extração de endereços}
\begin{itemize}
    \item Pré-processamento e processamento dos tweets: Corpus com \alert{414.637 palavras}, vocabulário com \alert{13.915 palavras}, comprimento máximo das sentenças do conjunto de dados com \alert{19 caracteres}.
    \item \alert{60.984} \textit{tweets} classificados manualmente.
    \item \alert{10.027} \textit{tweets} classificados manualmente em eventos de exceção e desse subconjunto foram encontrados \alert{8.112 endereços}. 
%    Desconsiderando o tipo de localidade APPROXIMATE (explicado mais adiante) --- (o que representa 80,90\% do total dos tweets classificados como eventos de exceção, sem considerar a classe Irrelevante).
\end{itemize}
\end{frame}
%------------------------------------------------
\begin{frame}{Resultado da distribuição das classes dos eventos de exceção do Corpus Twitter}
\begin{figure}[!htb]% H manda colocar exatamente nessa posição no texto (relativa aos parágrafos anterior e posterior)
	\centering
		\includegraphics[width=1\linewidth]{tweets_distribution_pt_.png}
	\label{fig:pizza_bus}
  %\source{Felipe Cordeiro Alves Dias, 2017}
\end{figure}
\end{frame}
%------------------------------------------------
\begin{frame}{Resultados dos modelos para classificação automatizada dos eventos de exceção}
\begin{table}[!htb]
\centering
\caption {Métricas das avaliações dos algoritmos utilizados para classificação dos \textit{tweets} em eventos de exceção}
\label {tab:metrics}
\begin{tabular}{c|c|c|c|c}
\toprule
\textbf{Algoritmo} & \textbf{ACC} & \textbf{PPV} & \textbf{TPR} & \textbf{\textit{f1-score}} \\
\midrule
\textit{Naive Bayes} Complementar & 0,941 & 0,949 & 0,941 & 0,944 \\
\hline
Árvore de Decisão & 0,965 & 0,965 & 0,965 & 0,965 \\
\hline
K-ésimo Vizinho mais Próximo & 0,970 & 0,971 & 0,970 & 0,970 \\
\hline
Regressão Logística & 0,969 & 0,968 & 0,969 & 0,968 \\
\hline
\alert{Perceptron multicamadas} & \textit{\alert{0,973}} & \textit{\alert{0,972}} & \textit{\alert{0,973}} & \textit{\alert{0,972}} \\
\hline
\textit{Naive Bayes} Multinomial & 0,953 & 0,952 & 0,953 & 0,949 \\
\hline
Floresta Aleatória & 0,970 & 0,970 & 0,970 & 0,970 \\
\hline
Máquina de Vetores de Suporte & 0,833 & 0,694 & 0,833 & 0,757 \\
\bottomrule
\end{tabular}
%\source{Felipe Cordeiro Alves Dias, 2019}
\end{table}
\end{frame}
%------------------------------------------------
\begin{frame}{Resultados da MC do modelo Perceptron multicamadas}
\begin{figure}[!htb]% H manda colocar exatamente nessa posição no texto (relativa aos parágrafos anterior e posterior)
	\centering
		\includegraphics[width=0.66\linewidth]{confusion_matrix_mlp_pt.png}
	\label{fig:pizza_bus}
  %\source{Felipe Cordeiro Alves Dias, 2017}
\end{figure}
\end{frame}
%------------------------------------------------
%\begin{frame}{Resultados da distribuição de endereços extraídos por classe}
%\begin{table}[!htb]
%\centering
%\caption {Quantidade\tnote{f} de endereços extraídos por classe}
%\label {tab:qtdExtractedAddresses}
%\begin{adjustbox}{max height=10mm, width=4in}
%\begin{threeparttable}
%\begin{tabular}{c|c|c|c|c|c}
%\toprule
%\textbf{Classe} & \textbf{\#endereços extraídos\tnote{a}} & \textbf{\textit{\#APP\tnote{b}}} & \textbf{\textit{\#GEO\tnote{c}}} & \textbf{\textit{\#RANGE\tnote{d}}} & \textbf{\textit{\#ROOF\tnote{e}}} \\
%\midrule
%Acidente & 3.439 & 7 & 805 & 1.130 & 1.497 \\
%\hline
%Irrelevante & 451 & 13 & 292 & 6 & 140 \\
%\hline
%Desastre Natural & 2.464 & 9 & 340 & 719 & 1.396 \\
%\hline
%Evento Social & 793 & 4 & 761 & 2 & 26 \\
%\hline
%Evento Urbano & 1.002 & 4 & 942 & 10 & 46 \\
%\midrule
%\midrule
%\textbf{Total} & 8.149 & 37 & 3.140 & 1.867 & 3.105 \\
%\bottomrule
%\end{tabular}
%\begin{tablenotes}
%\item[a] Total de endereços extraídos
%\item[b] Total de endereços extraídos com o tipo de localidade \textit{APPROXIMATE}
%\item[c] Total de endereços extraídos com o tipo de localidade \textit{GEOMETRIC\_CENTER}
%\item[d] Total de endereços extraídos com o tipo de localidade \textit{RANGE\_INTERPOLATED}
%\item[e] Total de endereços extraídos com o tipo de localidade \textit{ROOFTOP}
%\item[f] Total considerando endereços repetidos, a repetição é importante para identificarmos os endereços mais impactados por eventos de exceção.
%\end{tablenotes}
%\end{threeparttable}
%\end{adjustbox}
%\end{table}
%\end{frame}
%------------------------------------------------
%\begin{frame}{Resultados da distribuição de endereços extraídos por classe}
%Os \textit{tipos de localidades}\footnote{Disponível em \url{https://developers.google.com/maps/documentation/geocoding}. Acesso em 16 de setembro de 2018.} são classificados pela \textit{Google Geocoding API} em:
%\begin{enumerate}
%\item \textit{ROOFTOP} --- Indica que o resultado retornado há informações de localização com precisão a nível do endereço de rua.
%\item \textit{RANGE\_INTERPOLATED} --- Indica que o resultado retornado reflete uma aproximação interpolada entre dois pontos precisos (como interseções). Geralmente, os resultados interpolados são retornados quando os códigos geográficos do \textit{rooftop} não estão disponíveis para um endereço de rua.
%\item \textit{GEOMETRIC\_CENTER} --- Indica que o resultado retornado é o centro geométrico de um resultado.
%\item \textit{APPROXIMATE} --- Indica que o resultado retornado é aproximado.
%\end{enumerate}
%\end{frame}
%------------------------------------------------
\begin{frame}{Resultados da distribuição de endereços extraídos por classe}
\begin{enumerate}
\item \textit{Tweets} apenas com o ponto de interesse, ou seja, não consta explicitamente o endereço.
\item \textit{Tweets} sem informação de endereço.
\item \textit{Tweets} com nome de logradouro incomum (por exemplo \emph{passagem}, \emph{complexo viário}, \emph{ligação sentido}).
\item \textit{Tweets} com endereços com palavras concatenadas (por exemplo \emph{avenidapaulista}).
\end{enumerate}
\end{frame}
%------------------------------------------------
\begin{frame}{Resultado da análise visual da distribuição dos eventos de exceção na região central de São Paulo}
\begin{figure}[!htb]% H manda colocar exatamente nessa posição no texto (relativa aos parágrafos anterior e posterior)
	\centering
		\includegraphics[width=0.85\linewidth]{exception_events_sp.png}
	\label{fig:pizza_bus}
  %\source{Felipe Cordeiro Alves Dias, 2017}
\end{figure}
\end{frame}
%------------------------------------------------
\begin{frame}{Resultado dos endereços mais impactados}
    \begin{figure}[!htb]% H manda colocar exatamente nessa posição no texto (relativa aos parágrafos anterior e posterior)
	\centering
		\includegraphics[width=1\linewidth]{address_analysis_pt.png}
	\label{fig:pizza_bus}
  %\source{Felipe Cordeiro Alves Dias, 2017}
\end{figure}
\end{frame}
%------------------------------------------------
\begin{frame}{Resultados das linhas de ônibus mais impactadas por eventos de exceção}
\begin {table} [!htb]
\centering
\begin{adjustbox}{max height=10mm, width=4in}
\begin{threeparttable}
\caption {Linhas de ônibus mais impactadas por eventos de exceção\tnote{a}}
\label {tab:impacted_bus_code_lines}
\begin {tabular} {c|c|c}
 \toprule
\textbf{Código da linha} & \textbf{\# eventos de exceção} & \textbf{Letreiro} \\
    \midrule
    33389 & 1301  & TERM. PINHEIROS / METRÔ TUCURUVI  \\
\hline

    33284 & 1176  & ITAIM BIBI / METRÔ SANTANA  \\
\hline

    33121 & 1023  & TERM. PRINC. ISABEL / TERM. STO. AMARO  \\
\hline

    32805 & 1006  & TERM. PRINC. ISABEL / CHÁC. SANTANA  \\
\hline

    33112 & 933   & TERM. PQ. D. PEDRO II / JD. SÃO SAVÉRIO  \\
%\hline
%
%    33111 & 857   & TERM. AMARAL GURGEL / JD. DA SAÚDE  \\
%\hline
%
%    35229 & 841   & TURISMO / CIRCULAR  \\
%\hline
%
%    33443 & 816   & ANA ROSA / METRÔ SANTANA  \\
%\hline
%
%    32897 & 805   & LUZ / TERM. A. E. CARVALHO  \\
%\hline
%
%    35072 & 767   & METRÔ BARRA FUNDA / CONEXÃO PETRÔNIO PORTELA  \\
%\hline
%
%    32772 & 759   & TERM. PRINC. ISABEL / TERM. STO. AMARO  \\
%\hline
%
%    33253 & 754   & METRÔ BELÉM / JD. BONFIGLIOLI  \\
%\hline
%
%    33391 & 748   & METRÔ JABAQUARA / METRÔ SANTANA  \\
%\hline
%
%    32813 & 746   & PÇA. DA SÉ / CHÁC. SANTANA  \\
%\hline
%
%    32829 & 746   & TERM. BANDEIRA / TERM. CAPELINHA  \\
%\hline
%
%    34048 & 719   & LGO. SÃO FRANCISCO / JD. SELMA  \\
%\hline
%
%    33486 & 715   & TERM. PQ. D. PEDRO II / TERM. SÃO MATEUS  \\
%\hline
%
%    33236 & 708   & TERM. BANDEIRA / JD. JAQUELINE  \\
%\hline
%
%    33336 & 697   & PINHEIROS / IMIRIM  \\
%\hline
%
%    32816 & 693   & TERM. PQ. D. PEDRO II / TERM. STO. AMARO  \\
%\hline
%
%    33534 & 690   & CARDOSO DE ALMEIDA / MACHADO DE ASSIS  \\
%\hline
%
%    32838 & 647   & PÇA. DA SÉ / PQ. RES. COCAIA  \\
%\hline
%
%    33398 & 639   & CID. UNIVERSITÁRIA / METRÔ SANTANA  \\
\bottomrule
\end{tabular}
\begin{tablenotes}
            \item[a] Tabela completa no Apêndice D.
        \end{tablenotes}
\end{threeparttable}
\end{adjustbox}
\end{table}
\end{frame}
%------------------------------------------------
\begin{frame}{Considerações finais sobre a metodologia desenvolvida}
\begin{itemize}
    \item Uma nova metodologia para classificação de eventos de exceção e analisa seus respectivos impactos no sistema de transporte coletivo por ônibus da cidade de São Paulo.
    \item O algoritmo \alert{\textit{Multi-layer Perceptron}} obteve \alert{0,973 de acurácia} para classificação de \textit{tweets} em eventos de exceção.
    \item É possível extrair endereços de \textit{tweets} semi-estruturados usando apenas expressões regulares.
    \item A \alert{classificação automatizada} desses eventos é o primeiro passo para entender melhor como os eventos de exceção afetam a rede de transporte público.
    \item Metodologia aplicável em \alert{diferentes idiomas e cidades} (\alert{a GTFS é um formato ubíquo para o transporte público e ferramentas como a NLTK suporta vários idiomas}).
\end{itemize}
\end{frame}
%------------------------------------------------
\section{Caracterização do impacto dos eventos de exceção}
\begin{frame}
\Huge{\centerline{Caracterização do impacto}}
\Huge{\centerline{dos eventos de exceção}}
\end{frame}
%------------------------------------------------
\begin{frame}{Distribuição do número de eventos de exceção geolocalizados ao longo dos meses dos anos 12/15/16 e 17}
\begin{figure}[!htb]
	\centering
 	  %\caption{Distribuição do número de eventos de exceção geolocalizados}
		\includegraphics[width=1\linewidth]{geolocated_exception_events_distribution_pt_.png}
	\label{fig:geolocated_exception_events_distribution}
\end{figure}
\end{frame}
%------------------------------------------------
\begin{frame}{Distribuição das classes de eventos de exceção geolocalizados ao longo dos meses do ano de 2017}
\begin{figure}[!htb]
	\centering
 	  %\caption{Distribuição das classes de eventos de exceção geolocalizados ao longo dos meses do ano de 2017}
		\includegraphics[width=1\linewidth]{exception_events_classification_distribution_pt_.png}
	\label{fig:exception_events_classification_distribution}
\end{figure}
\end{frame}
%------------------------------------------------
\begin{frame}{Processo para correlação entre os dados AVL, GTFS e \textit{tweets} para análise do impacto dos eventos de exceção}
\begin{figure}[!htb]
	\centering
 	  \caption{Processo para correlação entre os dados AVL, GTFS e \textit{tweets} para análise do impacto dos eventos de exceção}
		\includegraphics[width=1\linewidth]{avl_tweets_correlation_pt.png}
	\label{fig:avl_tweets_correlation_pt}
\end{figure} 
\end{frame}
%------------------------------------------------
\begin{frame}{Equação utilizada para identificar velocidade mediana esperada}
    \begin{equation}
\label{eqVeloc}
\resizebox{.9 \textwidth}{!} 
{$
 f(n) =
  \begin{cases}
    0      & \quad \text{se  vel.~mediana~do~dia~do~evento} > \frac{\text{vel.~mediana~dos~dias~da~semana}}{\text{total~de~vel.~medianas}}\\
    1 & \quad \text{se vel.~mediana~do~dia~do~evento} \leq \frac{\text{vel.~mediana~dos~dias~da~semana}}{\text{total~de~vel.~medianas}}
  \end{cases}$
  }
\end{equation}
\end{frame}
%------------------------------------------------
\begin{frame}{Resultados da caracterização dos impactos em relação às velocidades medianas dos ônibus}
\begin{table}[!htb]
\centering
\caption {Porcentagem de ônibus dos grupos de linhas afetadas por eventos de exceção, a 1.000~m e 100~m de distância a partir dos pontos de parada, respectivamente, que tiveram a velocidade mediana reduzida nos meses do ano de 2017}
\label{tab:exceptEventVelocityImpAllStop}
\begin{adjustbox}{max height=10mm, width=3.5in}
\begin{tabular}{c|cc|cc|cc|cc}
\toprule
\textbf{Mês} & \multicolumn{2}{c}{\textbf{Acidente}} & \multicolumn{2}{c}{\textbf{Desastre Natural}} & \multicolumn{2}{c}{\textbf{Evento Social}} &
\multicolumn{2}{c}{\textbf{Evento Urbano}}\\
\cmidrule(l){2-3} \cmidrule(l){4-5} \cmidrule(l){6-7} \cmidrule(l){8-9}
 & 1.000 m & 100 m & 1.000 m & 100 m & 1.000 m & 100 m & 1.000 m & 100 m \\
\midrule
Janeiro & 83,33 &  100 & 
64,23 &  98,00 & 
100 & --- &
 100 & --- \\
\hline
Fevereiro & 70,58 &  100 &
 66,25 &  100 &
 100 & 100 &
 80 & --- \\
\hline
Março &  50,00 &  --- & 
66,66 &  100 &
85,00 & 100 &
68,18 & 100 \\
\hline
Abril & 87,50 &100 & 
 61,11 & 100 & 
 82,75 & 100 & 
 76,92 &  100 \\
\hline
Maio & 65,13 &  100 &
 58,82 &  100 &
 93,33 & 100 &
 50,00 & 100 \\
\hline
Junho & 54,46 &  100 &
 61,53 &  100 &
 76,47 & 100 &
 72,41 & 100 \\
\hline
Julho & 61,48 &  98,41 &
 66,66 & 100 &
 69,23 & 100 &
58,13 & 100 \\
\hline
Agosto & 57,86 & 87,17 &
 55,35 & 100 &
 85,54 & 100 & 
 68,10 & 90,90 \\
\hline
Setembro & 64,21 & 100 &
 42,10 & 100 &
 92,30 & 100 & 
 62,06 & 100 \\
\hline
Outubro & 70,49 & --- &
 56,81 & --- &
 80,00 & --- &
 61,11 & --- \\
\hline
Novembro & 66,66 & 100 &
 57,99 & 100 &
 92,85 & 100 &
 74,35 & 100 \\
\hline
Dezembro & --- & --- & --- & --- & --- & --- & --- & ---  \\
\midrule
\midrule
\textbf{\alert{Total}} & \alert{66,51} & \alert{98,39} & \alert{59,77} & \alert{99,80} & \alert{87,04} & \alert{100} & \alert{70,11} & \alert{98,86}  \\
\bottomrule
\end{tabular}
\end{adjustbox}
\end{table}
\end{frame}
%------------------------------------------------
\begin{frame}{Resultados da caracterização dos impactos em relação às velocidades medianas dos ônibus}
\begin{table}[!htb]
\centering
\caption {Porcentagem de impacto na velocidade média dos grupos de linhas afetadas por eventos de exceção a 1.000~m e 100~m de distância dos pontos de rota, respectivamente, nos meses do ano de 2017}
\label{tab:exceptEventVelocityImpAllShapes}
\begin{adjustbox}{max height=10mm, width=3.5in}
\begin{tabular}{c|cc|cc|cc|cc}
\toprule
\newline \textbf{Mês} & \multicolumn{2}{c}{\textbf{Acidente}} &
\multicolumn{2}{c}{\textbf{Desastre Natural}} & \multicolumn{2}{c}{\textbf{Evento Social}} &
\multicolumn{2}{c}{\textbf{Evento Urbano}}\\
\cmidrule(l){2-3} \cmidrule(l){4-5} \cmidrule(l){6-7} \cmidrule(l){8-9}
 & 1.000 m & 100 m & 1.000 m & 100 m & 1.000 m & 100 m & 1.000 m & 100 m \\
\midrule
Janeiro & 66,66 &  100 & 
 47,68 &  78,49 & 
 100 & 100 &
 100 & --- \\
\hline
Fevereiro & 35,29  &  100 &
 49,09 &  81,25 &
 100 & 100 &
 40,00 & 100 \\
\hline
Março  & 66,66  &  100 & 
 42,85 &  62,5 &
90,00 & 72,22 &
50,00 & 53,84 \\
\hline
Abril & 62,50 & 60,00 & 
47,05  & 100 & 
76,11 & 77,27 & 
89,47 &  90,90\\
\hline
Maio & 49,09 &  77,77 &
64,70 &  100 &
73,33 & 80,00 &
40,00 & 50,00 \\
\hline
Junho & 47,78 &  79,76 &
 46,15 &  70,00 &
 61,76 & 61,29 &
72,41 & 77,77 \\
\hline
Julho & 44,85  &  75,55 &
 66,66  & 83,33 &
48,14  & 75,00 &
41,86 & 61,53 \\
\hline
Agosto & 49,49 & 75,36 &
  44,44 & 71,42 &
  72,72 & 72,72 & 
70,00  & 56,75 \\
\hline
Setembro & 49,47  & 79,16 &
36,84  & 54,54 &
76,92  & 58,33 & 
55,17 & 73,91 \\
\hline
Outubro & 56,06 & 78,26 &
58,69  & 90,00 &
90,00  & 75,00 &
55,00 & 60,00 \\
\hline
Novembro & 54,32 & 66,66 &
 44,00 & 74,07 &
85,71  & 85,71 &
67,50  & 72,97 \\
\hline
Dezembro & --- & --- & --- & --- & --- & --- & --- & ---  \\
\midrule
\midrule
\textbf{\alert{Total}} & \alert{52,92} & \alert{81,13} & \alert{49,83} & \alert{78,69} & \alert{79,51} & \alert{77,95} & \alert{68,14} & \alert{69,76}  \\
\bottomrule
\end{tabular}
\end{adjustbox}
\end{table}
\end{frame}
%------------------------------------------------
\begin{frame}{Trabalhos relacionados a identificação de padrões de velocidade média dos dados AVL}
\begin{itemize}
    \item Identificar \alert{padrões relacionados a transferência (entre metrô e ônibus)}, por meio dos dados dos cartões inteligentes usados no transporte público da China.
    \item Identificar e apresentar visualmente \alert{padrões de movimentação humana}, no transporte público de Singapura. 
    \item Identificar \alert{padrões de rotas de táxi}, na cidade de Pequim, China.
    \item Identificar \alert{anomalias no comportamento do transporte público rodoviário da cidade do Rio de Janeiro}.
\end{itemize}
\end{frame}
%------------------------------------------------
\begin{frame}{Identificação de padrões de velocidade média dos dados AVL}
\begin{block}{}
        A proposta desse trabalho se diferencia das demais por encontrar os padrões de velocidade média existentes nos dados do transporte público por ônibus da cidade de São Paulo, considerando ainda a correlação com eventos de exceção extraídos de Redes Sociais.
\end{block}
\end{frame}
%------------------------------------------------
\begin{frame}{Resultados da identificação de padrões de velocidade média dos dados AVL}
\begin{table}[!htb]
\centering
\caption {Análise \textit{Apriori}\tnote{a} aplicada as velocidades médias (intervalos de 5 minutos) ao conjunto de dados AVL da SPTrans}
\label {tab:aprioriFull}
\begin{adjustbox}{max height=10mm, width=4in}
\begin{tabular}{c|c|c|c|c}
\toprule
\textbf{Mês} & \textbf{Regra de associação} & \textit{\textbf{Support}} & \textit{\textbf{Confidence}} & \textit{\textbf{Lift}} \\
\midrule
Fevereiro & 7 $\rightarrow$ 8 & 0,101 & 0,496 & 3,586\\
Abril & 7 $\rightarrow$ 8  & 0,108 & 0,456 & 3,188\\
Maio & 7 $\rightarrow$ 8 & 0,108 & 0,570 & 4,375\\
\midrule
Outubro & 8 $\rightarrow$ 7 & 0,100 & 0,595 & 3,433\\
Novembro & 8 $\rightarrow$ 7 & 0,104 & 0,446 & 3,369\\
\midrule
Janeiro & 11 $\rightarrow$ 12 & 0,137 & 0,476 & 1,729 \\
Junho & 11 $\rightarrow$ 12 & 0,129 & 0,632 & 1,656\\
Julho & 11 $\rightarrow$ 12 & 0,204 & 0,694 & 1,934\\
Agosto & 11 $\rightarrow$ 12 & 0,169 & 0,670 & 1,662\\
Outubro & 11 $\rightarrow$ 12 & 0,119 & 0,601 & 1,669\\
\bottomrule
\end{tabular}
\end{adjustbox}
\end{table}
\end{frame}
%------------------------------------------------
\begin{frame}{Resultados da identificação de padrões de velocidade média dos dados AVL}
\begin{table}[!htb]
\centering
\caption {(Continuação) Análise \textit{Apriori}\tnote{a} aplicada as velocidades médias (intervalos de 5 minutos) ao conjunto de dados AVL da SPTrans}
\label {tab:aprioriFull}
\begin{adjustbox}{max height=10mm, width=4in}
\begin{tabular}{c|c|c|c|c}
\toprule
\textbf{Mês} & \textbf{Regra de associação} & \textit{\textbf{Support}} & \textit{\textbf{Confidence}} & \textit{\textbf{Lift}} \\
\midrule
Fevereiro & 12 $\rightarrow$ 11 & 0,126 & 0,582 & 1,770\\
Março & 12 $\rightarrow$ 11 & 0,134 & 0,621 & 1,627\\
Abril & 12 $\rightarrow$ 11 & 0,123 & 0,601 & 2,013\\
Maio & 12 $\rightarrow$ 11 & 0,137 & 0,645 & 1,703\\
Setembro & 12 $\rightarrow$ 11 & 0,163 & 0,608 & 1,863\\
Novembro & 12 $\rightarrow$ 11 & 0,154 & 0,531 & 1,875\\
Dezembro & 12 $\rightarrow$ 11 & 0,143 & 0,432 & 2,073\\
\midrule
Fevereiro & 12 $\rightarrow$ 13 & 0,123 & 0,375 & 1,956\\
Março & 12 $\rightarrow$ 13 & 0,158 & 0,415 & 1,766\\
Junho & 12 $\rightarrow$ 13 & 0,141 & 0,370 & 1,907\\
\bottomrule
\end{tabular}
\end{adjustbox}
\end{table}
\end{frame}
%------------------------------------------------
\begin{frame}{Resultados da identificação de padrões de velocidade média dos dados AVL}
\begin{table}[!htb]
\centering
\caption {(Fim da continuação) Análise \textit{Apriori}\tnote{a} aplicada as velocidades médias (intervalos de 5 minutos) ao conjunto de dados AVL da SPTrans}
\label {tab:aprioriFull}
\begin{adjustbox}{max height=10mm, width=4in}
\begin{tabular}{c|c|c|c|c}
\toprule
\textbf{Mês} & \textbf{Regra de associação} & \textit{\textbf{Support}} & \textit{\textbf{Confidence}} & \textit{\textbf{Lift}} \\
\midrule
Abril  & 13 $\rightarrow$ 12 & 0,109 & 0,367 & 2,280\\
Maio & 13 $\rightarrow$ 12 & 0,161 & 0,425 & 1,942\\
Agosto & 13 $\rightarrow$ 12 & 0,147 & 0,366 & 1,830\\
Outubro & 13 $\rightarrow$ 12 & 0,150 & 0,417 & 1,737\\
\bottomrule
\end{tabular}
\end{adjustbox}
\end{table}
\end{frame}
%------------------------------------------------
\begin{frame}
\Huge{\centerline{Resultados das velocidades}}
\Huge{\centerline{médias inesperadas correlacionadas}}
\Huge{\centerline{aos eventos de exceção}}
\Huge{\centerline{Ref. aos \alert{pontos de parada}}}
\Huge{\centerline{Ao longo dos meses do ano de 2017}}
\end{frame}
%------------------------------------------------
\begin{frame}{Velocidades inesperadas relacionadas a acidentes}
\begin{figure}[!htb]
	\centering
 	  %\caption{Velocidades inesperadas dos ônibus impactados por eventos de exceção relacionados a acidentes a 100~m e 1.000~m dos pontos de parada, ao longo dos meses do ano de 2017}
		\includegraphics[width=1\linewidth]{apriori_analysis_stops_accidents_.png}
	\label{fig:apriori_analysis_stops_accidents}
\end{figure}
\end{frame}
%------------------------------------------------
\begin{frame}{Velocidades inesperadas relacionadas aos desastres naturais}
\begin{figure}[!htb]
	\centering
 	  %\caption{Velocidades inesperadas dos ônibus impactados por eventos de exceção relacionados a desastres naturais a 100~m e 1.000~m dos pontos de parada, ao longo dos meses do ano de 2017}
		\includegraphics[width=1\linewidth]{apriori_analysis_stops_natural_disasters_.png}
	\label{fig:apriori_analysis_stops_natural_disasters}
\end{figure}
\end{frame}
%------------------------------------------------
\begin{frame}{Velocidades inesperadas relacionadas aos eventos sociais}
\begin{figure}[!htb]
	\centering
 	  %\caption{Velocidades inesperadas dos ônibus impactados por eventos de exceção relacionados a eventos sociais a 100~m e 1.000~m dos pontos de parada, ao longo dos meses do ano de 2017}
		\includegraphics[width=1\linewidth]{apriori_analysis_stops_social_events_.png}
	\label{fig:apriori_analysis_stops_social_events}
\end{figure}
\end{frame}
%------------------------------------------------
\begin{frame}{Velocidades inesperadas relacionadas aos eventos urbanos}
\begin{figure}[!htb]
	\centering
 	  %\caption{Velocidades inesperadas dos ônibus impactados por eventos de exceção relacionados a eventos urbanos a 100~m e 1.000~m dos pontos de parada, ao longo dos meses do ano de 2017}
		\includegraphics[width=1\linewidth]{apriori_analysis_stops_urban_events_.png}
	\label{fig:apriori_analysis_stops_urban_events}
\end{figure}
\end{frame}
%------------------------------------------------
\begin{frame}
\Huge{\centerline{Resultados das velocidades}}
\Huge{\centerline{médias inesperadas correlacionadas}}
\Huge{\centerline{aos eventos de exceção}}
\Huge{\centerline{Ref. aos \alert{pontos de rota}}}
\Huge{\centerline{Ao longo dos meses do ano de 2017}}
\end{frame}
%------------------------------------------------
\begin{frame}{Velocidades inesperadas relacionadas a acidentes}
    \begin{figure}[!htb]
	\centering
 	  %\caption{Velocidades inesperadas dos ônibus impactados por eventos de exceção relacionados a acidentes a 100~m e 1.000~m dos pontos de rota, ao longo dos meses do ano de 2017}
		\includegraphics[width=1\linewidth]{apriori_analysis_shapes_accidents_.png}
	\label{fig:apriori_analysis_shapes_accidents}
\end{figure}
\end{frame}
%------------------------------------------------
\begin{frame}{Velocidades inesperadas relacionadas aos desastres naturais}
    \begin{figure}[!htb]
	\centering
 	  %\caption{Velocidades inesperadas dos ônibus impactados por eventos de exceção relacionados a desastres naturais a 100~m e 1.000~m dos pontos de rota, ao longo dos meses do ano de 2017}
		\includegraphics[width=1\linewidth]{apriori_analysis_shapes_natural_disasters_.png}
	\label{fig:apriori_analysis_shapes_natural_disasters}
\end{figure}
\end{frame}
%------------------------------------------------
\begin{frame}{Velocidades inesperadas relacionadas aos eventos sociais}
\begin{figure}[!htb]
	\centering
 	 % \caption{Velocidades inesperadas dos ônibus impactados por eventos de exceção relacionados a eventos sociais a 100~m e 1.000~m dos pontos de rota, ao longo dos meses do ano de 2017}
		\includegraphics[width=1\linewidth]{apriori_analysis_shapes_social_events_.png}
	\label{fig:apriori_analysis_shapes_social_events}
\end{figure}
\end{frame}
%------------------------------------------------
\begin{frame}{Velocidades inesperadas relacionadas aos eventos urbanos}
    \begin{figure}[!htb]
	\centering
 	  %\caption{Velocidades inesperadas dos ônibus impactados por eventos de exceção relacionados a eventos sociais a 100~m e 1.000~m dos pontos de rota, ao longo dos meses do ano de 2017}
		\includegraphics[width=1\linewidth]{apriori_analysis_shapes_urban_events_.png}
	\label{fig:apriori_analysis_shapes_urban_events}
\end{figure}
\end{frame}
%------------------------------------------------
\begin{frame}
\begin{table}[!htb]
\centering
\begin{adjustbox}{max height=10mm, width=3in}
\begin{threeparttable}
\caption {Análise \textit{Apriori} aplicada as velocidades médias (intervalos de 5 minutos) ao conjunto de dados AVL da SPTrans correlacionados aos eventos de exceção (a distância de 100~m\tnote{f} e 1.000~m\tnote{g}, respectivamente, dos pontos de parada de ônibus) dos meses do ano de 2017}
\label {tab:aprioriExceptFullStops}
\begin{tabular}{c|c|c|c|c|c}
\toprule
\begin{tabular}{@{}c@{}}\textbf{Classe}\\ \textbf{do evento}\end{tabular} & \begin{tabular}{@{}c@{}}\textbf{Total de}\\ \textbf{eventos}\tnote{a}\end{tabular}  & \begin{tabular}{@{}c@{}}\textbf{Total de Regras }\\ \textbf{de Associação}\tnote{b}\end{tabular}  & \textbf{Esperadas}\tnote{c} & \begin{tabular}{@{}c@{}}\textbf{Não}\\ \textbf{esperadas}\tnote{d}\end{tabular}  & \begin{tabular}{@{}c@{}}\textbf{Parcialmente}\\ \textbf{inesperadas}\tnote{e}\end{tabular} \\
\midrule
Acidente & 1.677 & 315.063 & 278.493 & 30.804 & 5.766 \\
\hline
\begin{tabular}{@{}c@{}}Desastre\\ Natural\end{tabular} &  912 & 115.301 & 99.206 & 14.282 & 1.813 \\
\hline
\begin{tabular}{@{}c@{}}Evento\\ Social\end{tabular} & 506 & 61.927 & 52.403 & 8.245 & 1.279 \\
\hline
\begin{tabular}{@{}c@{}}Evento\\ Urbano\end{tabular}  & 596 & 93.513 & 81.261 & 10.480 & 1.772 \\
\midrule
\textbf{\alert{Total}} & \alert{3.691} & \alert{585.804} & \alert{511.363} & \alert{63.811} & \alert{10.603} \\
\bottomrule
\toprule
\begin{tabular}{@{}c@{}}\textbf{Classe}\\ \textbf{do evento}\end{tabular} & \begin{tabular}{@{}c@{}}\textbf{Total de}\\ \textbf{eventos}\tnote{a}\end{tabular}  & \begin{tabular}{@{}c@{}}\textbf{Total de Regras }\\ \textbf{de Associação}\tnote{b}\end{tabular}  & \textbf{Esperadas}\tnote{c} & \begin{tabular}{@{}c@{}}\textbf{Não}\\ \textbf{esperadas}\tnote{d}\end{tabular}  & \begin{tabular}{@{}c@{}}\textbf{Parcialmente}\\ \textbf{inesperadas}\tnote{e}\end{tabular} \\
\midrule
Acidente & 3.029 & 3.980.542 & 3.415.780 & 385.728 & 179.034 \\
\hline
\begin{tabular}{@{}c@{}}Desastre\\ Natural\end{tabular} &  2.016 & 2.624.415 & 2.253.123 & 259.285 & 112.007 \\
\hline
\begin{tabular}{@{}c@{}}Evento\\ Social\end{tabular} & 764 & 1.262.805 & 1.118.546 & 100.224 & 44.035 \\
\hline
\begin{tabular}{@{}c@{}}Evento\\ Urbano\end{tabular}  & 980 & 1.481.040 & 1.296.476 & 125.803 & 58.761 \\
\midrule
\textbf{\alert{Total}} & \alert{6.789} & \alert{9.348.802} & \alert{8.083.925} & \alert{871.040} & \alert{393.837} \\
\bottomrule
\end{tabular}
\begin{tablenotes}
            \item[a] Total de eventos de exceção.
            \item[b] Total de correlações de velocidade média.
            \item[c] Regras esperadas ($Lift > 1$, $Support > 0,05$)
            \item[d] Regras de associação inesperadas ($Lift = 1$).
            \item[e] Regras de associação parcialmente inesperadas ($0 < Lift < 1$).
            \item[f] 3.545 eventos de exceção não atingiram linhas de ônibus no raio de 100~m.
            \item[g] 447 eventos de exceção não atingiram linhas de ônibus no raio de 1.000~m.
        \end{tablenotes}
\end{threeparttable}
\end{adjustbox}
\end{table}
\end{frame}
%------------------------------------------------
\begin{frame}
\begin{table}[!htb]
\centering
\begin{adjustbox}{max height=10mm, width=3in}
\begin{threeparttable}
\caption {Análise \textit{Apriori} aplicada as velocidades médias (intervalos de 5 minutos) ao conjunto de dados AVL da SPTrans correlacionados aos eventos de exceção (a distância de 100~m\tnote{g} e 1.000~m\tnote{h}, respectivamente, dos pontos de rota dos ônibus) dos meses do ano de 2017}
\label {tab:aprioriExceptFullShapes}
\begin{tabular}{c|c|c|c|c|c}
\toprule

%\begin{tabular}{@{}c@{}}Desastre\\ Natural\end{tabular} 
%\begin{tabular}{@{}c@{}}Evento\\ Social\end{tabular} 
%\begin{tabular}{@{}c@{}}Evento\\ Urbano\end{tabular} 

\begin{tabular}{@{}c@{}}\textbf{Classe}\\ \textbf{do Evento}\end{tabular} & \begin{tabular}{@{}c@{}}\textbf{Total de}\\ \textbf{Eventos}\tnote{b}\end{tabular}   & \begin{tabular}{@{}c@{}}\textbf{Qtd. Regras }\\ \textbf{de Associação}\tnote{c}\end{tabular}  & \textbf{Esperadas}\tnote{d} & \begin{tabular}{@{}c@{}}\textbf{Não}\\ \textbf{Esperadas}\tnote{e}\end{tabular} & \begin{tabular}{@{}c@{}}\textbf{Parcialmente}\\ \textbf{inesperadas}\tnote{f}\end{tabular}    \\
\midrule
Acidente & 2.367 & 3.390.690 & 3.164.726 & 171.860 & 54.104 \\
\hline
\begin{tabular}{@{}c@{}}Desastre\\ Natural\end{tabular} & 1.307 & 1.342.048 & 1.247.219 & 75.981 & 18.848 \\
\hline
\begin{tabular}{@{}c@{}}Evento\\ Social\end{tabular} & 704 & 1.522.423 & 1.433.700 & 67.835 & 20.888 \\
\hline
\begin{tabular}{@{}c@{}}Evento\\ Urbano\end{tabular}  & 825 & 1.602.343 & 1.499.305 & 79.155 & 23.883 \\
\midrule
\midrule \textbf{\alert{Total}} & \alert{5.203} & \alert{7.857.504} & \alert{7.344.950} & \alert{394.831} & \alert{117.723} \\
\bottomrule
\toprule
\begin{tabular}{@{}c@{}}\textbf{Classe}\\ \textbf{do evento}\end{tabular} & \begin{tabular}{@{}c@{}}\textbf{Total de}\\ \textbf{eventos}\tnote{a}\end{tabular}  & \begin{tabular}{@{}c@{}}\textbf{Total de Regras }\\ \textbf{de Associação}\tnote{b}\end{tabular}  & \textbf{Esperadas}\tnote{c} & \begin{tabular}{@{}c@{}}\textbf{Não}\\ \textbf{esperadas}\tnote{d}\end{tabular}  & \begin{tabular}{@{}c@{}}\textbf{Parcialmente}\\ \textbf{inesperadas}\tnote{e}\end{tabular} \\
\midrule
Acidente & 3.035 & 2.772.368 & 2.259.806 & 365.234 & 147.328 \\
\hline
\begin{tabular}{@{}c@{}}Desastre\\ Natural\end{tabular} & 2017 & 1.876.843 & 1.545.172 & 239.897 & 91.774 \\
\hline
\begin{tabular}{@{}c@{}}Evento\\ Social\end{tabular}  & 764 & 683.037 & 588.385 & 63.549 & 31.103 \\
\hline
\begin{tabular}{@{}c@{}} Evento\\ Urbano\end{tabular}  & 980 & 963.892 & 805.901 & 111.898 & 46.093 \\
\midrule
\midrule \textbf{\alert{Total}} & \alert{6.796} & \alert{6.296.140} & \alert{5.199.264} & \alert{780.578} & \alert{316.298} \\
\bottomrule
\end{tabular}
\begin{tablenotes}
            \item[a] Total de eventos de exceção.
            \item[b] Total de correlações de velocidade média.
            \item[c] Regras esperadas ($Lift > 1$, $Support > 0,05$)
            \item[d] Regras de associação inesperadas ($Lift = 1$).
            \item[e] Regras de associação parcialmente inesperadas ($0 < Lift < 1$).
            \item[f] 2.033 eventos de exceção não atingiram linhas de ônibus no raio de 100~m.
            \item[g] 440 eventos de exceção não atingiram linhas de ônibus no raio de 1.000~m.
        \end{tablenotes}
\end{threeparttable}
\end{adjustbox}
\end{table}
\end{frame}
%------------------------------------------------
\subsection{Considerações sobre a caracterização dos impactos dos eventos de exceção}\begin{frame}{Considerações sobre a caracterização dos impactos dos eventos de exceção}
\begin{itemize}
\item De acordo com os estudos realizados \alert{é possível caracterizar padrões inesperados e reduções de velocidades relacionadas aos eventos de exceção}. 
\item Tais padrões foram validados de acordo com os \alert{períodos de sazonalidade e dos eventos de exceção identificados nos \textit{tweets}}. 
\item \alert{Notícias veiculadas na mídia} correlacionadas aos padrões identificados:
\begin{itemize}
\item \alert{Greve geral, manifestações}.
\item \alert{Ataques aos ônibus de São Paulo}.
\item \alert{Ataques aos ônibus do Ceará}.
\end{itemize}

\end{itemize}
\end{frame}
%------------------------------------------------
\begin{frame}{Conclusão e contribuições}
\begin{itemize}
\item Estudo realizado para caracterização de eventos de exceção e de seus respectivos impactos no sistema de transporte público por ônibus da cidade de São Paulo, com dados reais obtidos de fontes públicas e heterogêneas:  \textit{tweets}, dados históricos dos módulos AVL e da GTFS.
\item Uma nova metodologia para extração e geolocalização dos endereços contidos nas publicações dos órgãos responsáveis por reportar eventos de exceção da cidade de São Paulo é proposta e validada.
\item Uma arquitetura distribuída para exploração e visualização de dados AVL.
\end{itemize}
\end{frame}
%------------------------------------------------
\begin{frame}{Conclusão e contribuições}
\begin{itemize}
\item Uma nova metodologia desenvolvida para \alert{extração e geolocalização automática de endereços a partir de mensagens postadas no Twitter}, adequada para as \alert{contas governamentais responsáveis pelas notificações de eventos de exceção da Cidade de São Paulo}.
\item  Modelos de \alert{classificação automatizada de eventos de exceção,  treinados com 60.984 \textit{tweets} classificados manualmente}.
\item A abordagem que utiliza as coordenadas espaciais dos \alert{pontos de parada de ônibus como referência} mostrou-se mais adequada do que a que usa os pontos de rota. Devido aos resultados semelhantes obtidos, \alert{menor custo computacional e margem de erro}.
\end{itemize}
\end{frame}
%------------------------------------------------
\begin{frame}{Trabalhos atuais}
\begin{block}{Trabalhos publicados}
DIAS, F. C. A.; CORDEIRO, D. \textit{Visualizing large datasets: A case study with data of the buses of São Paulo city}. \textit{In}: \textit{1st Workshop on the Distributed Smart City} (WDSC'2018), 2018, Salvador, BA. \textit{Proceedings of the 37th IEEE International Symposium on Reliable Distributed Systems}, 2018. p. 10-13.
\end{block}

\begin{block}{Trabalhos submetidos}
DIAS, F.C.A; CORDEIRO, D. \textit{Characterization of exception events and their respective impacts on the public transport system by bus of São Paulo}. Simpósio Brasileiro de Redes de Computadores e Sistemas Distribuídos (SBRC), 2019.
\end{block}
\end{frame}
%------------------------------------------------
\begin{frame}{Trabalhos futuros}

\begin{block}{Publicações}
\begin{itemize}
\item Uso dos \textit{tweets} na caracterização dos impactos dos eventos de exceção.
\item Identificação de padrões inesperados nas velocidades correlacionados aos eventos de exceção, por meio do Algoritmo \textit{Apriori}.
\item Revisão Sistemática.
\end{itemize}
\end{block}

\begin{itemize}
\item Implementar o fluxo de processamento de dados em \textit{streaming}, em um cenário de exploração e visualização de dados quase em tempo real.
\item Estabelecer uma cooperação entre a Acadêmia e a SPTrans para aplicação cotidiana dos estudos realizados por esse trabalho e outros relacionados a análise de grandes volumes de dados de transportes públicos.
\item Aplicar os estudos realizados por este trabalho a publicações de usuários que representam a sociedade civil.
\end{itemize}
\end{frame}
%------------------------------------------------
\begin{frame}
\titlepage % Print the title page as the first slide
\end{frame}
%------------------------------------------------
\end{document}